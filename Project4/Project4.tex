
\documentclass[12pt]{article}
\usepackage{graphicx}
\usepackage{amsmath}
\usepackage{float}
\usepackage[utf8x]{inputenc}
\usepackage{textcomp}
\usepackage{caption}
\usepackage{subcaption}
\title{Project 4}
\begin{document}
\maketitle
The energy of the simplest form of the Ising model is given by the following formula
\begin{equation}\label{eq::energy}
E = -J\sum_{<kl>}^{N}s_ks_l
\end{equation}
with $s_k = \pm 1$ represents spin up and spin down. 
The expectation value of the energy is given by:
\begin{equation}\label{eq::energy_exp}
<E> = \sum_i E_i P_i(\beta)
\end{equation}
where $Z$ is the partition function, and $P_i(\beta)$ is the probability of energy state $E_i$. This probability is given by the following formula.
\begin{equation}\label{eq::prob}
P_i(\beta) = \frac{e^{-\beta E_i}}{Z}
\end{equation} 
The partition function is given by 
\begin{equation}\label{eq::part_func}
Z = \sum_{i=1}^{N} e^{-\beta E_i}
\end{equation}
The expectation value for the magnetic moment is given by 
\begin{equation}\label{eq::mag_mom_exp}
<M> = \sum_i M_i P_i(\beta)
\end{equation}
where $N$ is the number of states. 
\\
\\
The expectation value of the absolute value of the magnetic moment is given by 

\begin{equation}\label{eq::abs(mag_mom_exp)}
<|M|> = \sum_i |M_i| P_i(\beta)
\end{equation}
In this project, we will use periodic boundary conditions, meaning that the spins in the end are bound to the spins in the other end. 

Using (\ref{eq::energy_exp}), the expectation value of a 4 spin system can be calculated (2x2-lattice) . From the lectures, we know that the partition function for this system is given by the following:
\begin{equation}
Z = \sum_{i=1}^{16}e^{-\beta E_i} = 2e^{+8J\beta} + 2e^{-8J\beta} + 12 = 4\cosh(8J\beta) + 12
\end{equation}
The partition function can also be calculated by first calculating the energies of each of the possible states, and then using (\ref{eq::part_func}) summing over all the possible states. 

The expectation value of the energy is then calculated from (\ref{eq::energy_exp}), and the following is obtained: 
\begin{equation}
<E> =-\frac{32\sinh(8J\beta) }{Z} 
\end{equation}
Next, we want to calculate the expectation value of absolute value of the magnetic moment. Using (\ref{eq::prob}) and (\ref{eq::abs(mag_mom_exp)})   we obtain 

\begin{equation}
<|M|> = \frac{8\cosh(8J\beta)+16}{Z}
\end{equation}

We also want to calculate the specific heat capacity of the system, which is given by the following general expression:
\begin{equation}
C_v = \frac{1}{kT^2}\frac{\partial(<E>)}{\partial\beta}
\end{equation}

This can also be written as 
\begin{equation}
C_v = \frac{1}{kT^2}\frac{<E^2>-<E>^2}{Z^2}
\end{equation}

The magnetic susceptibility, is defined as 
\begin{equation}\label{eq::susc1}
\chi = \frac{1}{kT^2}(<M^2>-<M>^2)
\end{equation}

In this case, we assume that $<M> = 0$, so we can modify (\ref{eq::susc1}) to 
\begin{equation}
\chi = \frac{1}{kT^2}(<M^2>)
\end{equation}












\end{document}