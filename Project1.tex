\documentclass[12pt]{article}
\usepackage{amsmath}
\usepackage{listings}
\lstset{language=C++}
\begin{document}
\title{Project 1}
\maketitle

\subsection{Introduction}

In this project we want to solve the one-dimensional Poisson equation with Dirichlet boundary conditions, using numerical methods. We want to find out how we can reduce the number of floating point operations, and so reduce the computation time, using more specialized  algorithms.  

We will compare three different methods for solving the Poisson equation, first an algorithm that solves the set of equations given any tridiagonal matrix. Then we use a method that is specialized to our specific case. Lastly we use the method of LU-decomposition. 

The Poisson equation in its general form is a  partial differential equation that describes many physical systems, amongst others it is an important equation in the field of electrostatics.

In this project we are working with a problem that has an analytical solution, although that is not always the case for the Poisson equation. 

The fact that we have an analytical solution makes it possible to check that the numerical results are actually correct. 

\subsection{Methods}
The general one-dimensional Poisson equation is given by 
\begin{equation}
-u''(x) = f(x)
\end{equation}
We want to solve this equation with respect to $u$ on the interval $x \in [0,1]$ and with boundary conditions $u(0) = u(1) = 0$
We are given the source term $f  = 100e^{(-10x)}$. Then the above differential equation
has a closed-form  solution given by $u(x) = 1-(1-e^{-10})x-e^{-10x}$
We  approximate the second
derivative of $u$ with 
\[
   -\frac{v_{i+1}+v_{i-1}-2v_i}{h^2} = f_i  \hspace{0.5cm} \mathrm{for} \hspace{0.1cm} i=1,\dots, n,
\]
where $f_i=f(x_i)$.
We can rewrite this as a set of equations on the form \[
   {\bf A}{\bf v} = \tilde{{\bf b}},
\]
where ${\bf A}$ is an $n\times n$  tridiagonal matrix which we write as 
\begin{equation}
    {\bf A} = \left(\begin{array}{cccccc}
                           2& -1& 0 &\dots   & \dots &0 \\
                           -1 & 2 & -1 &0 &\dots &\dots \\
                           0&-1 &2 & -1 & 0 & \dots \\
                           & \dots   & \dots &\dots   &\dots & \dots \\
                           0&\dots   &  &-1 &2& -1 \\
                           0&\dots    &  & 0  &-1 & 2 \\
                      \end{array} \right),
\end{equation}
and $\tilde{b}_i=h^2f_i$.
\\

This gives the following set:

\begin{align*}
2v_1 - v_2 &= \mathbf{\tilde{b}_1} \\
-v_1 + 2v_2 - v_3 &= \mathbf{\tilde{b}_2} \\
-v_2 + 2v_3 - v_4 &= \mathbf{\tilde{b}_3} \\
 &\vdots  \\
-v_{n-2} + 2v_{n-1} - v_{n} &= \mathbf{\tilde{b}_{n-1}}
\end{align*}


From this a general expression can be derived:


\begin{equation*}
-v_{i-1} + 2v_{i} - v_{i+1} = \mathbf{\tilde{b_i}} 
\end{equation*}


Substituting for $\mathbf{\tilde{b}_i}$ gives:


\begin{align*}
-v_{i-1} + 2v_{i} - v_{i+1} &= f_ih^2 \\
-\frac{(v_{i-1} - 2v_{i} + v_{i+1})}{h^2} &= f_i 
\end{align*}

\begin{equation*}
\end{equation*}
%We see that multiplying this vector with the matrix $A$ gives us %the equations for $f_i$ multiplied with $h^2$. 
We start by looking at a more general case, with a tridiagonal  matrix 

\begin{equation}
    {\bf A} = \left(\begin{array}{cccccc}
                           b_1& c_1 & 0 &\dots   & \dots &\dots \\
                           a_1 & b_2 & c_2 &\dots &\dots &\dots \\
                           & a_2 & b_3 & c_3 & \dots & \dots \\
                           & \dots   & \dots &\dots   &\dots & \dots \\
                           &   &  &a_{n-2}  &b_{n-1}& c_{n-1} \\
                           &    &  &   &a_{n-1} & b_n \\
                      \end{array} \right)\left(\begin{array}{c}
                           v_1\\
                           v_2\\
                           \dots \\
                          \dots  \\
                          \dots \\
                           v_n\\
                      \end{array} \right)
  =\left(\begin{array}{c}
                           \tilde{b}_1\\
                           \tilde{b}_2\\
                           \dots \\
                           \dots \\
                          \dots \\
                           \tilde{b}_n\\
                      \end{array} \right).
 \end{equation}            
 
   
 First, we want to develop a general algorithm that solves for $v$ given any tridiagonal matrix, with any diagonal elements $a_i, b_i, c_i$ \\
Arrays of length $N+2$ are initialized, one array for $a$-values one for the $b$-values, and one for the $c$-values. \\

An array of $x$-values is also initialized, with $x\in[0,1]$, and step length $$h=\frac{1}{N+1}$$. \\

The algorithm for the forward and backward substitution was implemented as follows: 
 \begin{lstlisting}[frame=single]
  for(int i = 2; i < N +2; i++){
            M[i] = M[i] - (B[i] * A[i-1])/M[i-1];
            F[i] = F[i] - (B[i] * F[i-1])/M[i-1];
        }
 
        U[N] = F[N]/M[N];
 
        for(int i = N -1 ; i > 1; i--){
            U[i] = (F[i] -  A[i] * U[i +1])/M[i] ;
        }
\end{lstlisting}

Here, $M$ is the array containing elements of the diagonal, $A$ is the vector containing  elements in the upper diagonal, and $B$ is the array containing  elements in the lower diagonal. The $F$ array is the array containing source term values, $f(x)$.

This algorithm will solve any set of equations that can be written as a product of a any tridiagonal matrix and a vector. However, for the matrix specific to this project, the elements along the diagonal are all identical, and the non diagonal elements are also all identical. This allows for developing a more efficient algorithm. 

  
 
\end{document}

 
