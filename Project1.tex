\documentclass[12pt]{article}
\usepackage{amsmath}
\begin{document}
\title{Project 1}
\maketitle

\subtitle{Introduction}

In this project we want to solve the one-dimensional Poisson equation with Dirichlet boundary conditions. 

The Poisson equation  in its general form is a  partial differential equation that describes many physical systems. 

In this project we will look at different ways of solving the one dimensional Poisson-equation numerically, and find out which numerical method that is the most efficient in terms of computer power needed. 

The general one-dimensional Poisson equation is given by 
\begin{equation}
-u''(x) = f(x)
\end{equation}
We want to solve this equation with respect to $u$ on the interval $x \in [0,1]$ and with boundary conditions $u(0) = u(1) = 0$

We are given $f  = 100e^-10x$ 

We  approximate the second
derivative of $u$ with 


\[
   -\frac{v_{i+1}+v_{i-1}-2v_i}{h^2} = f_i  \hspace{0.5cm} \mathrm{for} \hspace{0.1cm} i=1,\dots, n,
\]
where $f_i=f(x_i)$.



If $v$ is a vector on the form 


We can rewrite this as a set of equations on the form \[
   {\bf A}{\bf v} = \tilde{{\bf b}},
\]

where ${\bf A}$ is an $n\times n$  tridiagonal matrix which we rewrite as 
\begin{equation}
    {\bf A} = \left(\begin{array}{cccccc}
                           2& -1& 0 &\dots   & \dots &0 \\
                           -1 & 2 & -1 &0 &\dots &\dots \\
                           0&-1 &2 & -1 & 0 & \dots \\
                           & \dots   & \dots &\dots   &\dots & \dots \\
                           0&\dots   &  &-1 &2& -1 \\
                           0&\dots    &  & 0  &-1 & 2 \\
                      \end{array} \right),
\end{equation}
and $\tilde{b}_i=h^2f_i$.

\begin{equation*}
v = 

\left(\begin{array}{c}
                           v_1\\
                           v_2\\
                           \dots \\
                          \dots  \\
                          \dots \\
                           v_n\\
                      \end{array} \right)
\end{equation}

We see that multiplying this vector with the matrix $A$ gives us the equations for $f_i$ multiplied with $h^2$





We start by looking at a more general case, with a tridiagonal  matrix 

\begin{equation}
    {\bf A} = \left(\begin{array}{cccccc}
                           b_1& c_1 & 0 &\dots   & \dots &\dots \\
                           a_1 & b_2 & c_2 &\dots &\dots &\dots \\
                           & a_2 & b_3 & c_3 & \dots & \dots \\
                           & \dots   & \dots &\dots   &\dots & \dots \\
                           &   &  &a_{n-2}  &b_{n-1}& c_{n-1} \\
                           &    &  &   &a_{n-1} & b_n \\
                      \end{array} \right)\left(\begin{array}{c}
                           v_1\\
                           v_2\\
                           \dots \\
                          \dots  \\
                          \dots \\
                           v_n\\
                      \end{array} \right)
  =\left(\begin{array}{c}
                           \tilde{b}_1\\
                           \tilde{b}_2\\
                           \dots \\
                           \dots \\
                          \dots \\
                           \tilde{b}_n\\
                      \end{array} \right).
 \end{equation}            
 
   First, we want to develop a general algorithm that solves for $v$ given any tridiagonal matrix. 
   
   
Next, we use the fact that the matrix has identical matrix elements along the diagonal and identical values for the non-diagonal elements. 
 
\end{document}

 
