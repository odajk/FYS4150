\documentclass[12pt]{article}
\usepackage{braket}
\usepackage{amsmath}
\usepackage{listings}
\begin{document}
\title{Project 2}
\maketitle
\section{Method}



It can be shown that a unitary transformation of a vector conserves the dot product. 
The dot product between two of the basis vectors can be written as:

\begin{equation*}
\mathbf{v}_i^T \mathbf{v}_i=  \delta_{ij}
\end{equation*}
New vectors $\mathbf{w}_i$ are created by a unitary transformation of the basis vectors $\mathbf{v}_i$ 
\begin{equation*}
\mathbf{w}_i = \mathbf{Uv}_i
\end{equation*}
The dot product between the new vectors will be:
\begin{equation*}
\mathbf{w}_i^T\mathbf{w}_j  = (\mathbf{U}\mathbf{v}_i)^T\mathbf{U}\mathbf{v}_j = \mathbf{v}_i^T\mathbf{U}^T\mathbf{U}\mathbf{v}_j
\end{equation*}
Since $\mathbf{U}$ is a unitary matrix, 
\begin{equation*}
\mathbf{U}^T\mathbf{U} = \mathbf{1}
\end{equation*}
And so:
\begin{equation*}
\mathbf{w}_i^T\mathbf{w}_j  = \mathbf{v}_i^T \mathbf{v}_i=  \delta_{ij}
\end{equation*}
Thus, it has been proven that the dot product, and so also the orthogonality is conserved when the basis vectors $\mathbf{v}_i$ are acted on by a unitary transformation matrix. 
\\
\\
Furthermore, in a unitary transformation of a matrix equation, the eigenvalues are conserved, even though the eigenvectors are changed.  In Jacobi's method, a unitary transformation matrix transforms a matrix with arbitrary elements to a diagonal matrix, so after implementing Jacobi's method, it is easy to find the eigenvalues of the original equation. A more detailed explanation is given in the lecture notes of this course\footnote{Morten Hjorth-Jensen \textit{Computational Physics: Lecture Notes Fall 2015 }, 215-218} 
\subsection{Numerical implementation}


The following block of code applies Jacobi's method as long as the number of iterations is smaller than the maximum number of iterations and the maximum non-diagonal element is bigger than the tolerance, $\epsilon$. The maximum of iterations is put to $N^3$, where $N$ is the dimension of the matrices and $\epsilon = 10^{-8}$. 

%oda er dust og forstår ikke mappestrukturen sin, unnskyld git
\begin{lstlisting}
 while (num_iter <= max_iter && fabs(max) > epsilon)
        {
            max = max_elem(&i, &j, A, N);
            A = sim_transform(i, j, A, R,N);
            num_iter++;
        }
\end{lstlisting}
The \lstinline{max_elem}-function in the code above finds the maximum non-diagonal element of the matrix. As long as the absolute value of this element is not smaller than the tolerance, $\epsilon$, and the max number of iterations has not been reached, it will continue to loop, and calculate a new matrix using the \lstinline{sim_transform}-function. It is in this function that the Jacobi transformation is performed. The method was implemented as follows:
\newline
\begin{lstlisting}
   if (A(k, l) != 0.0){
        tau = (A(l,l) - A(k,k))/(2.0*A(k,l));

        if(tau > 0){
            t = 1.0/(tau + sqrt(1.0 + tau*tau));
            //t = +tau - sqrt(1+tau*tau);
        }
        else{
            t = -1.0/(-tau + sqrt(1.0 + tau*tau));
        }
        c = 1/(sqrt(1.0 +t*t));
        s = c*t;
    }
    else{
        c = 1.0;
        s = 0.0;
    }
    double a_kk, a_ll, a_ik, a_il, r_ik, r_il;
    a_kk = A(k,k);
    a_ll = A(l,l);
    A(k,k) = c*c*a_kk - 2.0*c*s*A(k,l) + s*s*a_ll;
    A(l,l) = s*s*a_kk + 2.0*c*s*A(k,l) + c*c*a_ll;
    A(k,l) = 0.0;  // hard-coding non-diagonal elements by hand
    A(l,k) = 0.0;  // same here
    for ( int i = 0; i < N+1; i++ ) {
      if ( i != k && i != l ) {
        a_ik = A(i,k);
        a_il = A(i,l);
        A(i,k) = c*a_ik - s*a_il;
        A(k,i) = A(i,k);
        A(i,l) = c*a_il + s*a_ik;
        A(l,i) = A(i,l);
      }
      r_ik = R(i,k);
      r_il = R(i,l);

      R(i,k) = c*r_ik - s*r_il;
      R(i,l) = c*r_il + s*r_ik;
\end{lstlisting}

\end{document}
